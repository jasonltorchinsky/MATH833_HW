\section{2}

Consider a one-dimensional linear model

\begin{equation}
	\dv{u}{t} = -a\,u + f + \sigma\,\dot{W},
\end{equation}

with $a = f = 1$, and $\sigma = 0.5$. Let the observational model be

\begin{equation}
	v = g\,u + \sigma^{o},
\end{equation}

where $\sigma^{o}$ is an independent zero-mean Gaussian random number at each observational time-instant. Let $g = 2$ and $\vr{\sigma^{o}} = 0.04$. Generate a time-series that has a length of 100 time-units with observation time-step size $\Delta t = 0.25$.

\begin{enumerate}[a)]
	\item Write down the standard Kalman filter to solve the posterior mean and posterior covariance.
	
	\item Write down a particle filter algorithm to solve the problem. Experiment with the number of particles in your simulation and explain if you need resampling.
\end{enumerate}
