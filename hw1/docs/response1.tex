\textit{Response.} 

\begin{enumerate}[(a)]

	\item To find the form of the condition distribution $\func{p}{\va{x}_1\ \vert\ \va{x}_2}$, we will use Bayes' theorem 

\begin{equation}
	\func{p}{\va{x}_1\ \vert\ \va{x}_2} = \frac{\func{p}{\va{x}_1,\ \va{x}_2}}{\func{p}{\va{x}_2}} = \frac{\func{p}{\va{x}}}{\func{p}{\va{x}_2}}.
\end{equation}

Writing the inverse of $\vb{R}$ as

\begin{equation}
	\vb{R}^{-1} = \mqty[ \vb{A} & \vb{B} \\ \vb{C} & \vb{D} ],
\end{equation}

we have

\begin{equation}
	\func{p}{\va{x}} = \frac{1}{\sqrt{\gpr{2\,\pi}^{n}\,\det{\vb{R}}}}\,e^{-\frac{1}{2}\,\gpr{\gpr{\va{x}_1 - \va{\mu}_1}^T\,\vb{A}\,\gpr{\va{x}_1 - \va{\mu}_1} + \gpr{\va{x}_1 - \va{\mu}_1}^T\,\gpr{\vb{B} + \vb{C}^T}\,\gpr{\va{x}_2 - \va{\mu}_2} + \gpr{\va{x}_2 - \va{\mu}_2}^T\,\vb{D}\,\gpr{\va{x}_2 - \va{\mu}_2}}}
\end{equation}

where $n = n_1 + n_2$ is the sum of the respective lengths of $\va{x}_1$ and $\va{x}_2$. The probability density function for $\va{x}_2$ is given by

\begin{equation}
	\func{p}{\va{x}_2} = \frac{1}{\sqrt{\gpr{2\,\pi}^{n_2}\,\det{\vb{R}_{22}}}}\,e^{-\frac{1}{2}\,\gpr{\va{x}_2 - \va{\mu}_2}^T\,\vb{R}_{22}^{-1}\,\gpr{\va{x}_2 - \va{\mu}_2}}.
\end{equation}

Therefore, by Bayes' theorem we have

\begin{align}
	\func{p}{\va{x}_1\ \vert\ \va{x}_2} & = \frac{\func{p}{\va{x}}}{\func{p}{\va{x}_2}} \nonumber \\
		&= \frac{1}{\sqrt{\gpr{2\,\pi}^{n_1}\,\frac{\det{\vb{R}}}{\det{\vb{R}_{22}}}}}\,e^{-\frac{1}{2}\,\gpr{\gpr{\va{x}_1 - \va{\mu}_1}^T\,\vb{A}\,\gpr{\va{x}_1 - \va{\mu}_1} + \gpr{\va{x}_1 - \va{\mu}_1}^T\,\gpr{\vb{B} + \vb{C}^T}\,\gpr{\va{x}_2 - \va{\mu}_2} + \gpr{\va{x}_2 - \va{\mu}_2}^T\,\gpr{\vb{D} - {\vb{R}_{22}}^{-1}}\,\gpr{\va{x}_2 - \va{\mu}_2}}} \nonumber \\
		&= \frac{1}{\sqrt{\gpr{2\,\pi}^{n_1}\,\frac{\det{\vb{R}}}{\det{\vb{R}_{22}}}}}\,e^{-\frac{1}{2}\,\gpr{\va{x} - \va{\mu}}^T\,\mqty[\vb{A} & \vb{B} \\ \vb{C} & \vb{D} - \vb{R}_{22}^{-1}]\,\gpr{\va{x} - \va{\mu}}} \label{eqn:pre_cond_dist}
\end{align}

To show that this equation represents a Gaussian distribution, we first calculate the ratio of determinants. Since $\vb{R}_{22}$ is invertible, we have

\begin{equation}
	\det{\vb{R}} = \det{\vb{R}_{11} - \vb{R}_{12}\,\vb{R}_{22}^{-1}\,\vb{R}_{21}}\,\det{\vb{R}_{22}},
\end{equation}

and therefore

\begin{equation}
	\frac{\det{\vb{R}}}{\det{\vb{R}_{22}}} = \det{\vb{R}_{11} - \vb{R}_{12}\,\vb{R}_{22}^{-1}\,\vb{R}_{21}} = \det{\overline{\vb{R}}}.
\end{equation}

Further, again from the invertibility of $\vb{R}_{22}$ we may write $\vb{R}^{-1}$ in terms of the blocks of $\vb{R}$

\begin{align}
	\vb{R}^{-1} &= \mqty[ \gpr{\vb{R}_{11} - \vb{R}_{12}\,\vb{R}_{22}^{-1}\,\vb{R}_{21}}^{-1} & -\gpr{\vb{R}_{11} - \vb{R}_{12}\,\vb{R}_{22}^{-1}\,\vb{R}_{21}}^{-1}\,\vb{R}_{12}\,\vb{R}_{22}^{-1} \\
	 -\vb{R}_{22}^{-1}\,\vb{R}_{21}\,\gpr{\vb{R}_{11} - \vb{R}_{12}\,\vb{R}_{22}^{-1}\,\vb{R}_{21}}^{-1} & \vb{R}_{22}^{-1}\,\vb{R}_{21}\,\gpr{\vb{R}_{11} - \vb{R}_{12}\,\vb{R}_{22}^{-1}\,\vb{R}_{21}}^{-1}\,\,\vb{R}_{12}\,\vb{R}_{22}^{-1} + \vb{R}_{22}^{-1} ] \nonumber \\
	 	 	&= \mqty[ \overline{\vb{R}}^{-1} & -\overline{\vb{R}}^{-1}\,\vb{R}_{12}\,\vb{R}_{22}^{-1} \\
	 				  -\vb{R}_{22}^{-1}\,\vb{R}_{21}\,\overline{\vb{R}}^{-1} & \vb{R}_{22}^{-1}\,\vb{R}_{21}\,\overline{\vb{R}}^{-1}\,\vb{R}_{12}\,\vb{R}_{22}^{-1} + \vb{R}_{22}^{-1} ]
\end{align}

Hence, the matrix in the exponent of Eqn.~\ref{eqn:pre_cond_dist} is given by

\begin{equation}
	\mqty[\vb{A} & \vb{B} \\ \vb{C} & \vb{D} - \vb{R}_{22}^{-1}] = \mqty[ \overline{\vb{R}}^{-1} & -\overline{\vb{R}}^{-1}\,\vb{R}_{12}\,\vb{R}_{22}^{-1} \\
	 				  -\vb{R}_{22}^{-1}\,\vb{R}_{21}\,\overline{\vb{R}}^{-1} & \vb{R}_{22}^{-1}\,\vb{R}_{21}\,\overline{\vb{R}}^{-1}\,\vb{R}_{12}\,\vb{R}_{22}^{-1} ]
\end{equation}


Therefore, by additionally utiling the symmetry of $\vb{R}$ (specifically $\vb{R}_{21} = \vb{R}_{12}^{T}$) and $\overline{\vb{R}}$, we may rewrite the exponent in Eqn.~\ref{eqn:pre_cond_dist} as

\begin{align}
	\gpr{\va{x} - \va{\mu}}^{T}\,\mqty[\vb{A} & \vb{B} \\ \vb{C} & \vb{D} - \vb{R}_{22}^{-1}]\,\gpr{\va{x} - \va{\mu}} &= \gpr{\va{x}_1 - \va{\mu}_1}^{T}\,\overline{\vb{R}}^{-1}\,\gpr{\va{x}_1 - \va{\mu}_1} \nonumber \\
			&\qquad - \gpr{\va{x}_1 - \va{\mu}_1}^{T}\,\gpr{\overline{\vb{R}}^{-1}\,\vb{R}_{12}\,\vb{R}_{22}^{-1} + \gpr{\vb{R}_{22}^{-1}\,\vb{R}_{21}\,\overline{\vb{R}}^{-1}}^{T}}\,\gpr{\va{x}_2 - \va{\mu}_2} \nonumber \\
			&\qquad - \gpr{\va{x}_2 - \va{\mu}_2}^{T}\,\vb{R}_{22}^{-1}\,\vb{R}_{21}\,\overline{\vb{R}}^{-1}\,\gpr{\va{x}_1 - \va{\mu}_1} \nonumber \\
			&\qquad + \gpr{\va{x}_2 - \va{\mu}_2}^{T}\,\vb{R}_{22}^{-1}\,\vb{R}_{21}\,\overline{\vb{R}}^{-1}\,\vb{R}_{12}\,\vb{R}_{22}^{-1}\,\gpr{\va{x}_2 - \va{\mu}_2} \nonumber \\
		&= \gpr{\va{x}_1 - \va{\mu}_1}^{T}\,\overline{\vb{R}}^{-1}\,\gpr{\va{x}_1 - \va{\mu}_1} \nonumber \\
			&\qquad - \gpr{\va{x}_1 - \va{\mu}_1}^{T}\,\overline{\vb{R}}^{-1}\,\gpr{\overline{\va{\mu}} - \va{\mu}_1} \nonumber \\
			&\qquad - \gpr{\overline{\va{\mu}} - \va{\mu}_1}^{T}\,\overline{\vb{R}}^{-1}\,\gpr{\va{x}_1 - \va{\mu}_1} \nonumber \\
			&\qquad + \gpr{\overline{\va{\mu}} - \va{\mu}_1}^{T}\,\overline{\vb{R}}^{-1}\,\gpr{\overline{\va{\mu}} - \va{\mu}_1} \nonumber \\
		&= \gpr{\va{x}_1 - \overline{\va{\mu}}}^{T}\,\overline{\vb{R}}^{-1}\,\gpr{\va{x}_1 - \overline{\va{\mu}}}.
\end{align}

Therefore, Eqn.~\ref{eqn:pre_cond_dist} may be written in the form

\begin{align}
	\func{p}{\va{x}_1\ \vert\ \va{x}_2} &= \frac{1}{\sqrt{\gpr{2\,\pi}^{n_1}\,\det{\overline{\vb{R}}}}}\,e^{-\frac{1}{2}\,\gpr{\va{x} - \overline{\va{\mu}}}^T\,\overline{\vb{R}}^{-1}\,\gpr{\va{x} - \overline{\va{\mu}}}},
\end{align}

which implies the conditional distribution $\func{p}{\va{x}_{1}\ \vert\ \va{x}_2}$ is Gaussian with mean $\overline{\va{\mu}}$ and covariance $\overline{\vb{R}}$.

	\item Source code is available from the GitHub repository
	
\begin{center}
	\url{https://github.com/jasonltorchinsky/MATH833_HW}
\end{center}

	In short, the code takes in an input parameter \texttt{-{}-N} (or \texttt{-n}) to use for $N$. It then generates $N$ samples from a two-dimensional joint Gaussian distribution with the above mean and covariances using NumPy's \texttt{multivariate\_normal} function. Of these, we select the samples with $0.9 < x_2 < 1.1$ and report their mean and variance, along with the $95\%$ confidence interval given by
	
\begin{equation}
	\gbkt{\widehat{X}_n - z_c\,\sqrt{\frac{s_n^2}{n}},\ \hat{X}_n + z_c\,\sqrt{\frac{s_n^2}{n}}}
\end{equation}

where $n$ is the number of samples from the conditional distribution, $\widehat{X}_n$ is the mean of the samples from the conditional distribution, $s_n^2$ is the varaince of the samples from the conditional distribution, and $z_c = 1.96$ corresponds to the $95\%$ confidence interval.

Although the results of any given run vary, the results for a single run of the reported $N$ are given in Table~\ref{tbl:prob1_num_res}, and should be compared to the analytic mean $0.5$ and analytic variance $0.75$ of the conditional distribution.

\begin{table}[H]
	\centering
	\begin{tabular*}{\textwidth}{c| @{\extracolsep{\fill}}c @{\extracolsep{\fill}}c @{\extracolsep{\fill}}c}
		\hline
		$N$    & Numerical Mean & Numerical Variance & Confidence Interval \\ \hline
		$10^3$ & 0.5786         & 0.7040             & $\gbkt{0.3680,\ 0.7891}$ \\
		$10^4$ & 0.5209         & 0.6288             & $\gbkt{0.4505,\ 0.5914}$ \\
		$10^5$ & 0.4979         & 0.7412             & $\gbkt{0.4737,\ 0.5220}$
	\end{tabular*}
	\caption{ Numerical results for Problem 1. }
	\label{tbl:prob1_num_res}
\end{table}

\end{enumerate}